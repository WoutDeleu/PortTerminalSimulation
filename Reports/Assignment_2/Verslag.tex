\documentclass[]{article}

\usepackage[numbers]{natbib}

%opening
\usepackage{float}
\usepackage{hyperref}
\usepackage[utf8]{inputenc}
\usepackage{graphicx}
\usepackage{listings}
\usepackage{color}
\usepackage{tocloft}
%\renewcommand{\cftpartleader}{\cftdotfill{\cftdotsep}} % for parts
%\renewcommand{\cftchapleader}{\cftdotfill{\cftdotsep}} % for chapters
\renewcommand{\cftsecleader}{\cftdotfill{\cftdotsep}} % for sections, if you really want! (It is default in report and book class (So you may not need it).
\setlength{\parskip}{16pt}

\title{Project Assignment Part II:\@ Simulation of a base scenario }

\author{Jef Jacobs \\ Toon Eeraerts \\ Wout Deleu}
\date{Semester 2}

\begin{document}
\setlength{\parindent}{0pt} \maketitle \tableofcontents \newpage %geen indent bij nieuwe paragraaf

\section{Introduction}
During this assignment, a base was implemented to build a simulation for the
`Yard storage assignment problem'. The basic structure was designed and there
is a basic scenario implemented. During this report, the basics design choices
will be discussed, as well as the results of the simulation.

The basic technology used is Python. The code is written as dynamic as
possible, using global variables and booleans to variate parameters and the
overall flow of the simulation. The simulation is based on a discrete event
simulation, with the possibility of online simulation.

To refresh the environment in brief, the goal is to simulate a yard, in which
codokterbriefkenntainergroups\footnote{Containers arrive only in group. Containers
	individually are (for now) not looked at individually. This means they can't be
	split up, they are stored in the same place, they enter and leave the yard at
	the same time.} come and gow. They can arrive from vessels or from trucks and
trains. The goal is to simulate the storage situations in the yard, as a result
of the in and out flow.

\section{Simulation parameters}
The first and maybe most primary parameter to know is the amount of simulations
that needs to be run to get significant results. This can be calculated by the
following formula: $$ JEF HEEFT EEN KLEINE PENIS $$

\subsection{Generation of parameters}
In order to emulate the basic behavior of the yard, there needs to be a stream
of containergroups, with their own properties. To perform an online simulation,
this stream needs to be generated on the fly. In the prior assignment, a study
was held to figure out distributions, which will be used to generate the
necessary information regarding the containergroups. This is necessary to
generate the input to feed the online simulation.

The generation of the groups happens at random times, which are calculated
using the arrival time of the previous group, and a random interval time. This
interval time is being generated using a .... \textbf{TODO:: UITLEGGEN -
	distributie}

When the arrival time of a batch of containers is determined, some properties
of that container group needs to be generated. The properties are: 
\begin{itemize}
	\item The type of containers
	\item The number of containers in the group
	\item The service time needed
	\item The arrival position of the group
	\item The departure position of the group
	\item The flow type (which can be import or export)
\end{itemize}

The type of containers is in this case randomly selected with a weight of 69\%
in favor of the \textit{normal} containers. The other 31\% is designated to the
\textit{reefer} containers. Each group has a certain amount of containers in
them. This is chosen based on .... \textbf{TODO:: UITLEGGEN - distributie}.
Each instance has a service time, which is the time containers needs to stay in
the yard before further actions are taken. This factor is directly dependent on
the flow type. If it is import or export, the service time is by default 48
hours. But if it is stated as a transhipment (which is technically a subtype of
export), than it can vary between .... \textbf{TODO:: UITLEGGEN \-
	distributie}.

The arrival position or berthing location (\textbf{TODO:: Klopt dit??} )of a
vessel is randomly selected .... \textbf{TODO:: UITLEGGEN \- distributie}. The
same principle holds for the departure position.

\subsection{Evaluation of results \Large TODO::: EEN ANDERE TITEL}
Each simulation run will store different characteristics which, will be used to
evaluate the performance of the simulation. The most important characteristics
are: \begin{itemize}
	\item Amount of rejected containers and container groups (total and for each type
	      individually)
	\item The total and average travel distance
	\item Per YardBlock:
	      \begin{itemize}
		      \item The maximal occupancy at any given tim
		      \item The maximal occupancy per day (\textbf{= average occupancy per day})
	      \end{itemize}
	\item Average daily occupancy over all containers: The average occupancy per day over
	      all the YardBlocks \[\forall i \in days: \frac{\sum_{(x \in YardBlocks)}
			      \overline{Occupancy_{i,x}}}{\#YardBlocks}\]
\end{itemize}

\section{Results}

\subsection{Basic scenario}
The basic scenario that is discussed in this report describes a decision rule
which stores every containergroup that arrives, if there is space in the yard.
If there is no space, the container is rejected. This will be referred to as
\textit{FIFO (First In First Out)}. This apply's to arrival of containers, and
wether or not they are kept in the yard. FIFO says that the first container
that arrives, has a priority over the ones which arrive after the first one. If
there is no space for an arriving containergroup, it will be rejected. While
the next containergroup arrives, the check for space will happen again, and so
on. 

Two different approaches to block assignment are implemented. The block
assignment rule has affect on which block is chosen to store a containergroup.
The two different situations studied here are \textit{arrival based} and
\textit{departure based}. Arrival based and departure based both are both based
on the minimal distance between 2 points. The yard block chosen is the closest
block to the arrival point, in case of arrival based, or closest to the
departure point, in case of departure based.
\subsection{Departure based}

\begin{table}[h]
	\centering
	\begin{tabular}{|c|c|}
		\hline
		Containers Rejected            & 0.0                  \\ \hline
		CG Rejected                    & 0.0                  \\ \hline
		Normal Rejected                & 0.0                  \\ \hline
		Reefer Rejected                & 0.0                  \\ \hline
		Total Travel Distance          & 7471576.566740074    \\ \hline
		AVG Travel Distance Containers & 467.63038278646087   \\ \hline
		AVG daily total Occupancy      & 0.012189307994940618 \\ \hline
	\end{tabular}
	\caption{FIFO departure based statistics}
\end{table}

\subsubsection{Arrival based}

\begin{table}[h]
	\centering
	\begin{tabular}{|c|c|}
		\hline
		Containers Rejected            & 0.0                  \\ \hline
		CG Rejected                    & 0.0                  \\ \hline
		Normal Rejected                & 0.0                  \\ \hline
		Reefer Rejected                & 0.0                  \\ \hline
		Total Travel Distance          & 6069403.996520092    \\ \hline
		AVG Travel Distance Containers & 380.0611989034411    \\ \hline
		AVG daily total Occupancy      & 0.011198756687437594 \\ \hline
	\end{tabular}
	\caption{...}
\end{table}
\subsection{Stressing the system}
\end{document}