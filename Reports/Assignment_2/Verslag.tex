\documentclass[]{article}

\usepackage[numbers]{natbib}

%opening
\usepackage{float}
\usepackage{hyperref}
\usepackage[utf8]{inputenc}
\usepackage{graphicx}
\usepackage{listings}
\usepackage{color}
\usepackage{tocloft}
%\renewcommand{\cftpartleader}{\cftdotfill{\cftdotsep}} % for parts
%\renewcommand{\cftchapleader}{\cftdotfill{\cftdotsep}} % for chapters
\renewcommand{\cftsecleader}{\cftdotfill{\cftdotsep}} % for sections, if you really want! (It is default in report and book class (So you may not need it).
\setlength{\parskip}{16pt}

\title{Reflectie}

\author{Wout Deleu}
\date{Semester 2}

\begin{document}
\setlength{\parindent}{0pt} \maketitle \tableofcontents \newpage %geen indent bij nieuwe paragraaf

\section{Introductie}
Gedurende deze persoonlijke reflectie zal ik mijn mening en gevoel geven rond
deze cursus ethiek en de bijkomende opdrachten. We kregen al eerder vakken die
ietwat filosofisch in insteek waren, zoals wijsbegeerte en religie. Ik kan in
principe de focus leggen op de inhoud en invulling van dit vak, maar ik verkies
persoonlijk om iets meer de focus te leggen op het grotere plaatje, namelijk
filosofische vakken in een richting zoals Industriële Ingenieurswetenschappen.

Ik wil de mening echter wel op voorhand even kort kaderen, aangezien ik zelf
student ben in de richting Industriële Ingenieurswetenschappen. Deze richting
draait rond definities, zwart-wit-redeneringen en formules. Wetenschap in een
erg toegepaste vorm. Daarom heb ik ook het gevoel dat heel wat van mijn
medestudenten heel weinig voeling en interesse vertonen in reflectieve vakken.
Persoonlijk denk ik wel dat ik het hier relatief makkelijk mee heb, ik heb
enkele mensen in mijn dichte vriendenkring die wijsbegeerte studeren, waar ik
altijd wel een goeie babbel mee kan hebben. Maar in het algemeen ben ik een
persoon die vaak reflecteert, en dingen in vraag durft te stellen. Dit kan gaan
over persoonlijke dingen, maar regelmatig ook over maatschappij-overstijgende
dingen, zoals een tijd geleden gedurende een hardloopsessie, over de
klimaatopwarming. Hierbij dacht ik na over de waarom er niet op gehandeld
wordt, en welke omstandigheden noodzakelijk zouden moeten zijn om een aanzet te
geven tot handelen.

\section{Plaats van deze vakken binnen Industriële Ingenieurswetenschappen}
Zoals al vermeld ligt de focus van deze richting op toegepaste wetenschap. Dit
houdt in dat de meeste thema's die aan bod komen als correct worden gesteld. Er
is hierbij weinig ruimte voor de filosofische/ethische aspecten. Deze kunnen
kort aan bod komen, maar dit is vaak maar een verwaarloosbaar stuk van de
lectuur. Ook zijn de aangeleerde vaardigheden `juist', omdat ze op de een of
andere manier bewezen zijn. Dit staat meestal niet ter discussie.
Daarbijkomend, als een student een discussie start over een mogelijks
alternatieve oplossing, dan moet kunnen aangetoond dat deze oplossing op een
manier beter is dan de geziene. Dit moet kunnen bewezen worden. Dit staat, naar
mijn mening, ietwat haaks op de eerder beschouwde filosofische en spirituelere
vakken. Om een vanzelfsprekend voorbeeld te geven, in de cursus Ethiek wordt
een sectie gewijd aan de de definitie en nuances van het begrip
professionalisme. Iets wat in mijn ogen voor de hand liggend is, maar vanaf je
er begint over na te denken, blijkt toch dat er veel nuanceringen mogelijk
zijn.

Maar net omdat dit in vele van de vakken in de opleiding geen tijd nemen voor
deze kwesties, vind ik het des te belangrijker dat er wel bij stil gestaan
wordt. Het stimuleert een kritische geest op aspecten die voorafgaand niet eens
als een onderwerp zouden worden gezien. Ik vind het zeker noodzakelijk dat de
veronderstelde waarheden die aangeleerd worden in vraag te stellen. Of meer
algemeen, de tools aan te rijken om deze in vraag te kunnen stellen. Deze
kritische mindset is erg belangrijk op veel facetten van het leven.

\nocite{*} \bibliographystyle{plain}
\bibliography{bibfile.bib}
\end{document}